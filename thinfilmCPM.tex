\documentclass[%
aps, pre,
amsmath,amssymb,
reprint,%
%author-year,%
% author-numerical,%
%superscriptaddress,
%showpacs, showkeys,
]{revtex4-1}

\usepackage{trinhcmd}
\usepackage{microtype}
\usepackage{graphicx}% Include figure files
\usepackage{rotating}
\usepackage{dcolumn}% Align table columns on decimal point
\usepackage{bm}% bold math
\usepackage{color}
% \usepackage{amsmath, amsthm, amssymb}
\usepackage{upgreek}
\usepackage{natbib}
\usepackage[inactive, tightpage]{preview}

\DeclareMathOperator{\Div}{div}
\DeclareMathOperator{\trace}{trace}
\DeclareMathOperator{\cp}{cp}
\DeclareMathOperator{\ecp}{ecp}
\DeclareMathOperator{\sd}{sd}
\DeclareMathOperator{\curl}{curl}
\DeclareMathOperator{\id}{id}
\DeclareMathOperator{\codim}{codim}
\DeclareMathOperator{\spa}{span}
\DeclareMathOperator{\argmin}{\arg\min}

\newcommand{\surf}{\mathcal{S}}
\newcommand{\band}{B(\surf)}

\newcommand{\lap}{\Delta}
\newcommand{\grad}{\nabla}
\newcommand{\slap}{\lap_{\surf}}
\newcommand{\sgrad}{\grad_{\surf}}
\newcommand{\sDiv}{\Div_{\surf}}
\newcommand{\sInt}{\int\limits_{\surf}}

\newcommand{\dx}{h}
\newcommand{\dt}{\tau}

\newcommand{\Real}{\mathbb{R}}

\newcommand{\todo}[1]{\textbf{\textcolor{blue}{TODO: #1}}}
\newcommand{\tens}[1]{\mathbf{#1}}

\newtheorem{theo}{Theorem}[section]
\newtheorem{defi}[theo]{Definition}
\newtheorem{principle}[theo]{Principle}
\newtheorem{cor}[theo]{Corollary}
\newtheorem{lem}[theo]{Lemma}
\newtheorem{prop}[theo]{Proposition}

\newcommand{\ea}{\mb{e}_1}
\newcommand{\eb}{\mb{e}_2}
\newcommand{\ec}{\mb{e}_3}

\renewcommand{\ol}[1]{\overline{#1}}



\begin{document}

\preprint{Draft Version}

\title[Thin films on curved surfaces]{Thin film flows on general curved surfaces with gravity and surface tension}

\author{Colin B.~Macdonald}%
\affiliation{Oxford Centre for Collaborative and Applied Mathematics, Mathematical Institute, University of Oxford, Oxford OX2 6GG, UK}
\author{Thomas M\"{a}rz}%
\affiliation{Oxford Centre for Collaborative and Applied Mathematics, Mathematical Institute, University of Oxford, Oxford OX2 6GG, UK}
\author{Philippe H. Trinh}%
\thanks{Corresponding author: trinh@maths.ox.ac.uk}
\affiliation{Oxford Centre for Industrial and Applied Mathematics, Mathematical Institute, University of Oxford, Oxford OX2 6GG, UK}

\date{\today}

\begin{abstract}
We present a numerical study of gravity-driven thin film flows on curved surfaces. We are particularly motivated by the aspect of gravity-driven fingering instabilities. 
\end{abstract}


\pacs{}% PACS
\keywords{thin film dynamics, closest point method, surface-intrinsic differential equations}%Use showkeys class option if keyword
\maketitle

\section{Introduction}\label{sect:Intro}

There is much interest in the study of thin film dynamics on general curved surfaces. In comparison with problems on flat surfaces, the main challenge is to propose a convenient three-dimensional coordinate system for its study. Even once such a coordinate system has been established, there are notable challenges, particularly in regards to surfaces with removable coordinate singularities.

Much of the research is based on surfaces that are easily parameterized: cylinders, tori. 

Two physical problems of interest:
\begin{enumerate}
\item Work of Mayo et al on gravity-driven instabilities
\item Work of Trinh, Lister, etc. on Rayleigh-Taylor instability
\end{enumerate}

\section{Thin film equations}

\subsection{Parameterization and non-dimensionalization}

For clarity and completeness, we recapitulate the essential details behind the derivation of the lubrication equations on general surfaces. Our equations follow from Howell \cite{howell_2003}, with additional details found in \cite{myers_2002} and \cite{roy_2002}. 

Let us consider thin film flow on a two-dimensional surface, $\mathcal{S} \subset \mathbb{R}^3$. Every point on the surface is characterized by a normal vector $\mb{n}$, and two principal curvatures, denoted $\kappa_1$ and $\kappa_2$, with $\kappa_1 \leq \kappa_2$. We define a mapping from the Cartesian $(x,y,z)$ system to an orthogonal curvilinear coordinate system given by $(s_1, s_2, n)$, where $s_1$ and $s_2$-curves are lines of constant principal curvature, and $n$ is distance measured in the normal direction from the surface. If $\mb{e}_1$, $\mb{e}_2$, and $\mb{e}_3$ denote the unit orthogonal vectors for the coordinate system, then a general point in space is given by the position vector
\begin{equation}
\mb{r}(s_1, s_2, n) = \mb{r}_c(s_1, s_2) + n \mb{e}_3, 	
\end{equation}
%
where the surface is given by $\mb{r}_c = \mb{r}_c(s_1, s_2)$. In order to properly move between Cartesian and curvilinear derivatives, we require the surface scale factors, defined by $a_k = \lvert \pd{\mb{r}_c}{s_k}\rvert$, and spatial scale factors, $m_k = \lvert \pd{\mb{r}}{q_k}\rvert$. It then follows that $m_1 = a_1(1 - \kappa_1 n)$, $m_2 = a_2(1 - \kappa_2 n)$, and $m_3 = 1$. With these scale factors, formulae for vector operators can be found from [---]. 

We now non-dimensionalize, with the key consideration being that the scaling of the thin film is small compared to the scaling along the substrate. We will dimensionalize the problem so that the pressure, $p$, balances the free surface curvature, $\scr{K}$. The scaling on the velocities come from balancing the viscous term with the pressure term in the momentum equations. The scalings are:
\begin{gather}
[\kappa_1] = [\kappa_2] = 1/a, \ [p] = \gamma/a, \\
[s_1] = [s_2] = L, \ [n] = \delta L, \\ 
[u_1] = [u_2] = \frac{\delta^2 \gamma L}{\mu a}, \ [u_3] = \frac{\delta^3 \gamma L}{\mu a}, \\ 
[t] = \frac{\mu a}{\delta^2 \gamma}. 
% \\ [\sigma_{ij}] = \frac{\gamma}{a}.
\end{gather}

The key consideration to follow is in regards to how the length scale, $L$ of $s_1$ and $s_2$, compare to the substrate curvature $1/a$. We use these scalings and relabel all the variables to be non-dimensional. Notice that under this non-dimensionalisation, the scaling factors become $m_1 = a_1 + \Oh(\delta L/a)$, $m_2 = a_2 + \Oh(\delta L/a)$, and $m_3 = 1$.

\subsection{Lubrication equations}

We consider incompressible flow of fluid, given by a velocity field $\mb{u} = u_1 \ea + u_2 \eb + u_3 \ec$. We assume the fluid possesses a free surface, given by $n = h(x_1, x_2, t)$. On the substrate, the fluid satisfies no-flux condition and zero-slip conditions, $\mb{u} = 0$, on $n = 0$.	

On the free surface, $n = h(s_1, s_2, t)$, the fluid satisfies the kinematic and dynamic conditions, 
\begin{gather}
u_3 = \pd{h}{t} + \frac{u_1}{m_1} \pd{h}{x_1} + \frac{u_2}{m_2} \pd{h}{x_2} \\
\mb{n} \cdot (\sigma_{ij} n_j) =  -\gamma \scr{K}, \quad \text{and} \quad 
\mb{t} \cdot (\sigma_{ij} n_j) =  0 
\end{gather}
%
where $\mb{t}$ is the tangent and $\mb{n} = [n_1, n_2, n_3]$, the normal, at a point on the surface, $n = h(x_1, x_2, t)$, $\sigma = (\sigma)_{ij}$ is the stress tensor, given in (\ref{stress}), and $\scr{K}$ is twice the mean curvature.

We integrate the continuity equation, $\nabla \cdot \mb{u} = 0$, from $n = 0$ to $n = h$, and combine with the no-flux and kinematic condition to give the following conservation equation for the flux
\begin{equation}
m_1 m_2 \pd{h}{t} + \pd{}{x_1}\left[\int_0^h m_2 u_1 \, \de{n}\right] + \pd{}{x_2}\left[\int_0^h m_1 u_2 \, \de{n}\right] = 0,
\end{equation}
%
on $n = h$. Turning to the dimensional momentum equations, the relevant terms once the thin film limit is applied will only involve the following terms
\begin{subequations} \label{stokes}
\begin{align}
\frac{1}{a_k} \pd{p}{x_k} &= \pdd{u_k}{n} + \rho g (\hat{\mb{g}} \cdot \mb{e}_k) + \ldots, \qquad k = 1, 2\label{stokes12} \\
\pd{p}{n} &= \rho g (\hat{\mb{g}} \cdot \mb{e}_3) + \ldots \label{stokes3}
\end{align}
\end{subequations}

Next, we reduce the momentum equations. Assuming that $\delta^2 \Rey = \rho[u] L/\mu \ll 1$, we are left with the usual Stokes equations that include the bond number, 
\[
B = \frac{\rho g L^2}{\mu [u_1]} = \frac{\rho g La}{\gamma}
\]

\noindent Performing the analogous scalings for the rest of the momentum equations and using the reduction of the scale factors \eqref{scalereduce} gives
\begin{subequations} \label{stokes}
\begin{align}
\frac{1}{a_k} \pd{p_k}{x_k} &= \pdd{u_k}{n} + B \mb{g} \cdot \mb{e}_k + \Oh(\delta^2), \qquad k = 1, 2\label{stokes12} \\
\pd{p_3}{n} &= \Oh(\delta B, \delta^2). \label{stokes3}
\end{align}
\end{subequations}

\noindent \emph{Remark about signs of curvature}: Notice that depending on whether the thin film is on the outside or inside of the substrate, the defined curvatures may be negative. We will always assume that $a > 0$ and all length scales in \eqref{scalings} to be positive, \emph{e.g.} flow on the outside or inside of a circular cylinder has curvature $\kappa = \mp 1/a$. 

\subsection{Boundary conditions}

In order to derive the reduced free-surface conditions, we expand the normal $\mb{n}$, tangential vector, $\mb{t}$, and stress components $\sigma_{ij}$ in powers of the aspect ratio, $\delta$. After work, it can be seen that the two tangential stress conditions are
\begin{align}
\pd{u_1}{n} + \left(\frac{\delta L}{a}\right) \frac{u_1 a_1 \kappa_1}{\tilde{m_1}} &= \Oh(\delta), \\ 
\pd{u_2}{n} +  \left(\frac{\delta L}{a}\right) \frac{u_2 a_2 \kappa_2}{\tilde{m_2}} &= \Oh(\delta).
\end{align}

The normal stress condition, $\mb{n} \cdot \mb{F}$, is 
\begin{equation} \label{presfree}
-p = \scr{K} + \Oh(\delta^2) \qquad \text{on $n = h(x_1, x_2, t)$},
\end{equation}

\noindent where the full curvature is given in \eqref{curv}.In fact, the curvature can be expanded:
\begin{equation} \label{curvereduce}
\scr{K} = (\kappa_1 + \kappa_2) + \frac{\delta a}{L} \nabla_s^2 h + \frac{\delta L}{a} (\kappa_1^2 + \kappa_2^2) + \Oh(\delta^2)
\end{equation}

\noindent and $\nabla_s$ is the surface Laplacian with respect to the substrate:
\begin{equation}
\nabla_s^2 = \frac{1}{a_1 a_2} \left[ \pd{}{x_1} \left( \frac{a_2}{a_1} \pd{}{x_1}\right) + \pd{}{x_2} \left(\frac{a_1}{a_2} \pd{}{x_2}\right)\right].
\end{equation}


\newpage

The elements of the non-dimensionalized stress tensor, which should be scaled with $\gamma/a$ are given by \eqref{stress}, or  
\begin{subequations} \label{stressre}
\begin{align}
\sigma_{11} &= -p + \Oh(\delta^2)  \\ 
\sigma_{22} &= -p + \Oh(\delta^2) \\ 
\sigma_{33} &= -p + \Oh(\delta^2) \\ 
\sigma_{12} &= \frac{\delta^2}{m_1 m_2} \left[ m_1 \pd{u_1}{x_2} - \pd{m_1}{x_2} u_1 + m_2 \pd{u_2}{x_1} - \pd{m_2}{x_1} u_2 \right] \\ 
\sigma_{13} &= \delta \pd{u_1}{n} - \delta^2 \left(\frac{L}{a}\right) \frac{u_1}{m_1} \pd{m_1}{n} + \Oh(\delta^3) \\ 
\sigma_{23} &= \delta \pd{u_2}{n} -  \delta^2 \left(\frac{L}{a}\right) \frac{u_2}{m_2} \pd{m_2}{n} + \Oh(\delta^3).
\end{align}
\end{subequations}

\noindent where we note that $\partial m_i/\partial n = -a_i \kappa_i$. 

The surface is given by $n = h(x_1, x_2, t)$, or with $\mb{r}(x_1, x_2, n) = \mb{r}_c(x_1, x_2) + h(x_1, x_2, t) \ec$. Recalling the definitions of the scale factors \eqref{hkdef2}, we have two tangent vectors at the surface:
\begin{align*}
\tilde{\mb{t}}_1 = \pd{\mb{r}(x_1,x_2,h)}{x_1} &= \tilde{h}_1 \ea + \delta \pd{h}{x_1} \ec, \\ 
\tilde{\mb{t}}_2 = \pd{\mb{r}(x_1,x_2,h)}{x_2} &= \tilde{h}_2 \ea + \delta \pd{h}{x_2} \ec,
\end{align*}

\noindent where
\begin{equation}
\tilde{h}_i = a_i \left(1 - \left(\tfrac{\delta L}{a} \right) \kappa_i h \right)
\end{equation}

\noindent are the scale factors evaluated on the surface. Notice that at this point we will not make any assumption on the size of $\delta L/a$ except that it is at most $\Oh(1)$. Henceforth, we use the notation of tilde to indicate surface quantities. 

The outwards normal at the surface is given by 
\begin{equation}
\tilde{\mb{n}} = \frac{1}{c} \left[ -\frac{\delta }{m_1} \pd{h}{x_1} \ea - \frac{\delta }{m_2} \pd{h}{x_2} \eb + \ec\right],
\end{equation}

\noindent where $c = |\nabla (\delta n- \delta h)|$.

The force on the free surface, $\mb{F} = \sigma_{ij} n_j$ is given by 
\begin{equation}
\begin{split}
\mb{F} &= \frac{1}{c}\left[\left(\frac{1}{\tilde{m_1}}\pd{h}{x_1}p + \pd{u_1}{n} \right) \delta 
+ \left(\frac{\delta^2 L}{a}\right) \frac{u_1 a_1 \kappa_1}{\tilde{m_1}} + \Oh(\delta^3)\right]\ea \\ 
&\qquad + \frac{1}{c}\left[\left(\frac{1}{\tilde{m_2}}\pd{h}{x_2}p + \pd{u_2}{n} \right) \delta 
+ \left(\frac{\delta^2 L}{a}\right) \frac{u_2 a_2 \kappa_2}{\tilde{m_2}} + \Oh(\delta^3)\right]\eb \\ 
&\qquad \qquad + \frac{1}{c}\biggl[ -p + \Oh(\delta^2) \biggr] \ec
\end{split}
\end{equation}

\noindent The two tangential stress conditions are $\mb{t}_1 \cdot \mb{F} = 0 = \mb{t}_2 \cdot \mb{F}$, and so 
\begin{align}
\pd{u_1}{n} + \left(\frac{\delta L}{a}\right) \frac{u_1 a_1 \kappa_1}{\tilde{m_1}} &= \Oh(\delta), \\ 
\pd{u_2}{n} +  \left(\frac{\delta L}{a}\right) \frac{u_2 a_2 \kappa_2}{\tilde{m_2}} &= \Oh(\delta).
\end{align}

\noindent The reason why we keep the $\delta L/a$ term is because in cases where there the surface and substrate curvatures are very different, this term will become $\Oh(1)$. 

The normal stress condition, $\mb{n} \cdot \mb{F}$, is 
\begin{equation} \label{presfree}
-p = \scr{K} + \Oh(\delta^2) \qquad \text{on $n = h(x_1, x_2, t)$},
\end{equation}

\noindent where the full curvature is given in \eqref{curv}.In fact, the curvature can be expanded:
\begin{equation} \label{curvereduce}
\boxed{\scr{K} = (\kappa_1 + \kappa_2) + \frac{\delta a}{L} \nabla_s^2 h + \frac{\delta L}{a} (\kappa_1^2 + \kappa_2^2) + \Oh(\delta^2)}
\end{equation}

\noindent and $\nabla_s$ is the surface Laplacian with respect to the substrate:
\begin{equation}
\nabla_s^2 = \frac{1}{a_1 a_2} \left[ \pd{}{x_1} \left( \frac{a_2}{a_1} \pd{}{x_1}\right) + \pd{}{x_2} \left(\frac{a_1}{a_2} \pd{}{x_2}\right)\right].
\end{equation}

\noindent Notice that with the approximation \eqref{curvereduce}, the free-surface curvature is only a function of $(x_1, x_2, t)$, \emph{i.e.} the position on the substrate. 

\subsection{Thin film PDE}

We integrate the Stokes equations \eqref{stokes} to give 
\begin{equation}
u_k = \left(\frac{1}{a_k} \pd{p}{x_k} - B\mb{g} \cdot e_k\right)\left[  \frac{n^2}{2} - hn \right],
\end{equation}
%
and use this in the conservation equation to give
\begin{equation}
\pd{h}{t} - \nabla_s \cdot \left\{ \frac{h^3}{3}  \left[ \left( \frac{1}{a_1} \pd{p}{x_1} - B\mb{g} \cdot \ea\right)\ea  
+ \left( \frac{1}{a_2} \pd{p}{x_2} - B\mb{g} \cdot \eb\right)\eb \right]\right\}.
\end{equation}

\noindent We can now use the connection between pressure and curvature in \eqref{presfree} to obtain
\begin{equation} \label{thinfilm}
\boxed{\pd{h}{t} + \nabla_s \cdot \left\{ \frac{h^3}{3} \left[\nabla_s \scr{K} + B\Bigl( (\mb{g} \cdot \ea)\ea + (\mb{g} \cdot \eb)\eb\Bigr) \right]\right\} = 0}
\end{equation}

\noindent where 
\[
\nabla_s \scr{K} = \pd{\scr{K}}{x_1} \frac{\ea}{a_1} + \pd{\scr{K}}{x_2} \frac{\eb}{a_2}.
\]

\noindent Notice that \eqref{thinfilm} is the governing equation for the thin film that only depends on the surface gradients, $\nabla_s$. Although we have originally defined the surface gradients in terms of the lines of curvature coordinates, $x_1$ and $x_2$, we are now free to use another parameterization of the surface, and simply re-write the new surface gradient in terms of this parameterization. 

%
%Whose components are given in \eqref{stress}, and where $f_{ij} = \Oh(1)$ (note that the stress tensor is symmetric, but we have kept the indices as so for clarity). We write the normal and tangental vectors in the form
%\begin{align}
%\mb{n} &= \delta b_1 \ea + \delta b_2 \eb + \ec, \\ 
%\mb{t}_1 &= \delta b_2 \ea - \delta b_1 \eb, \\
%\mb{t}_2 &= \ea - \delta b_1 \ec.
%\end{align}
%
%\noindent where $b_1, b_2 = \Oh(1)$ and are given on \eqref{surfnorm}. 

%\begin{gather}
%\sigma_{11} = \sigma_{22} = \sigma_{33} = -p + \Oh(\delta^2) \\
%\sigma_{12} = \Oh(\delta^2) \\ 
%\sigma_{13} = \sigma_{23} = \Oh(\delta).
%\end{gather}

\newpage

\section{The closest point method} \label{sec:cpm}
The closest point method, introduced in \cite{Ruuth/Merriman:jcp08:CPM}, is an embedding technique for solving PDEs posed on embedded surfaces.
The central idea is to extend functions and differential operators to the surrounding space (here $\Real^3$) and to solve an embedding equation which is
a 3D-analog of the original equation. This is appealing since the extended versions of the operators $\sgrad$, $\sDiv$, and $\slap$
in the closest point framework turn out to be $\grad$, $\Div$, and $\lap$.
Hence, embedding equations are accessible to standard finite difference techniques and existing codes can be reused. 

\todo{Explain surface diff ops}

\subsection{Surface Representation}
The closest point method utilizes the \textit{closest point representation} of a surface which is given in terms of the closest point function
\begin{align}\label{eqn:cp_function}
	\cp(x) = \arg \min\limits_{\hat{x} \in \surf} |x - \hat{x}|.
\end{align}
For a point $x$ in the embedding space, $\cp(x)$ is the point on the surface $\surf$ which is closest in Euclidean distance to $x$.
This function is well-defined in a tubular neighborhood or narrow band $\band$ of the surface and is as smooth as the underlying surface \cite{Maerz/Macdonald:cpfunctions}.

The closest point function can be derived analytically for many common surfaces.
When a parameterization or triangulation of the surface is known, the closest point function can be computed numerically \cite{Ruuth/Merriman:jcp08:CPM}.

\todo{illustration of $\cp$}

\subsection{Extending Functions and Differential Operators}
Using the closest point representation, we can extend values of a surface function $u:\surf \to \Real$ 
into the surrounding band $\band$ by defining $\bar{u} : \band \to \Real$ by
\begin{align}\label{eqn:cpext}
  \bar{u}(x) &:= u(\cp(x)).
\end{align}
Notably, $\bar{u}$ will be constant in the direction normal to the surface and this property is key to the closest point method:
it implies that an application of a Cartesian differential operator to $\bar{u}$
is equivalent to applying the corresponding intrinsic surface differential operator to $u$.
We state this as principles below; these mathematical principles were established in \cite{Ruuth/Merriman:jcp08:CPM} and proven in \cite{Maerz/Macdonald:cpfunctions}.
\begin{principle} \label{thm:gradient_principle}
	\emph{(Gradient Principle):} Let $\surf$ be a surface embedded in $\Real^n$ and let $u$ be a function, defined on $\Real^n$,
	that is constant along directions normal to the surface, then
	\begin{align}
		\grad u(x) &= \sgrad u(x) \qquad \forall \; x \in \surf. 
	\end{align}
\end{principle}
\begin{principle} \label{thm:divergence_principle}
	\emph{(Divergence Principle):} Let $\surf$ be a surface embedded in $\Real^n$. If $\vec{j}$ is a vector field on $\Real^n$ that is
	tangent to $\surf$ and tangent to all surfaces displaced a fixed Euclidean distance from $\surf$, then
	\begin{align}
		\Div \vec{j}(x) &= \sDiv \vec{j}(x) \qquad \forall \; x \in \surf.
	\end{align}
\end{principle}

Direct consequences of these principles are
\begin{align}
	\sgrad u(x) &= \grad \bar{u}(x), \label{4D_first} \\
	\sDiv \vec{j}(x) &= \Div \bar{\vec{j}}(x), \qquad \forall \; x \in \surf, \label{4D_second}
\end{align}
where $\bar{\vec{j}}$ is the closest point extension of a tangential flux $\vec{j}$.
Moreover, since $\grad \bar{u}$ is tangential to level-surfaces of the Euclidean distance-to-$\surf$ map \cite{Ruuth/Merriman:jcp08:CPM,Maerz/Macdonald:cpfunctions}, 
combining Principles~\ref{thm:gradient_principle} and \ref{thm:divergence_principle} yields
\begin{align}
	\slap u(x) &= \lap \bar{u}(x), \label{4D_third} \\
	\sDiv \left( g \sgrad u \right)(x) &= \Div \left( \bar{g} \grad \bar{u} \right)(x), \qquad \forall \; x \in \surf, \label{4D_third2}
\end{align}
where $g:\surf \to \Real$ is a scalar diffusivity and $\bar{g}$ its closest point extension.

Additionally, if $\tens{G}$ is a diffusion tensor which maps to the tangent space, i.e. the corresponding flux $\vec{j} = \tens{G} \sgrad u$ is tangential, then
Principles \ref{thm:gradient_principle} and \ref{thm:divergence_principle} also imply that
\begin{align}
	\sDiv \left( \tens{G} \sgrad u \right)(x) &= \Div \left( \bar{\tens{G}} \grad \bar{u} \right)(x), \qquad \forall \; x \in \surf, \label{4D_fourth}
\end{align}
where $\bar{\tens{G}}$ is the closest point extension of $\tens{G}$.

Finally, we note that a closest point extension $\bar{u}$ is characterized \cite{vonGlehn:mol} by
\begin{align}
	\bar{u} &= \bar{u} \circ \cp,
\end{align}
i.e. it is the closest point extension of itself.



%\subsection{Embedding Equation and Discretization}
\subsection{Gaussian Diffusion}
We start with the Gaussian diffusion equation on a closed surface $\surf$ in order to demonstrate the embedding idea:
% \begin{subequations}
% \begin{equation}
% \pd{u}{t} = \slap u, \qquad x \in \surf, \; t > 0, \label{4B_first} 
% \end{equation}
% \end{subequations}

\begin{subequations}
\begin{align}
\partial_t u &= \slap u, \qquad x \in \surf, \; t > 0, \label{4B_first} \\
u|_{t=0} &= u^0. \label{4B_second}
\end{align}
\end{subequations}  

Using \eqref{4D_third} we obtain the following embedding problem
\begin{subequations}
	\begin{align}
		& \partial_t v = \lap v && x \in \band, \; t > 0, \label{4C_first} \\
		& v|_{t=0} = u^0 \circ \cp \label{4C_second} \\
		& v = v \circ \cp && x \in \band, \; t > 0 \label{4C_third} \;.
	\end{align}
\end{subequations}  
Here \eqref{4C_second} says that we start the process with a closest point extension of the initial data $u^0$, while
condition \eqref{4C_third} guarantees that $v$ stays a closest point extension for all times and hence we can rely on the principles
which give us \eqref{4C_first} as the 3D-analog of \eqref{4B_first}. 


In order to cope with condition \eqref{4C_first} Ruuth \& Merriman \cite{Ruuth/Merriman:jcp08:CPM} suggested the following semi-discrete (in time) iteration:
after initialization $v^0 = u^0 \circ \cp$, alternate between two steps
\begin{subequations}
	\begin{align}
		&1.\text{ Evolve} & \quad w &= v^n + \dt \lap v^n, &&&&&&&&\\
		&2.\text{ Extend} & \quad v^{n+1} &= w \circ \cp,
	\end{align}
\end{subequations}
where $\dt$ is the time-step size. 
Here step 1 evolves \eqref{4C_first} of the embedding problem, while step 2 reconstructs the surface function or rather its closest point extension
to make sure that \eqref{4C_third} is satisfied at time $t_{n+1}$. 

A fully discrete version of the Ruuth \& Merriman iteration needs a discretization of the spatial operators in step 1 
and an interpolation scheme in step 2. Using a uniform Cartesian grid in $\Real^3$
the Laplacian in our example can be discretized with the standard 7 point finite difference stencil.
The interpolation scheme in step 2 is necessary since the data $w$ is given only on grid points and $\cp(x)$ is hardly ever a grid (even though $x$ is).
In order to get around that we interpolate the data $w$ with tri-cubic interpolation and extend the interpolant $W$ rather than $w$:
\begin{align}
	v^{n+1}_i &= W \circ \cp(x_i)
\end{align}
where $x_i$ is the grid point corresponding to the $i$-th component of the array $v^{n+1}$. Since tri-cubic interpolation is linear in the data $w$, this
operation can also be implemented as an extension matrix $E$ acting on the 1D array $v^{n+1}$. The fully discrete Ruuth \& Merriman iteration reads then
\begin{subequations}
	\begin{align}
		&1.\text{ Evolve} & \quad w &= v^n + \dt L v^n, &&&&&&&&\\
		&2.\text{ Extend} & \quad v^{n+1} &= E w.
	\end{align}
\end{subequations}

The computation is performed in a \emph{computational band} for two reasons: firstly, the closest point function is defined in the band $\band$ 
hence the computational band must be contained in $\band$. Secondly, the code can be made more efficient by working on a narrow surrounding the surface $\surf$.
A nice feature of the Ruuth \& Merriman iteration is that no artificial boundary conditions on the boundary of the band need to be imposed.  
This has to do with the extension step: the grid points are overwritten at each time step with the value at their closest points.
Note also, that no artificial boundary conditions are imposed in the embedding problem.

For the sake of efficiency optimization of the width of the computational band is reasonable. 
The bandwidth depends on the degree of the interpolant and on the finite difference stencil used. 
Suppose we use an interpolant of degree $p$ and are working in $d$-dimensions, this requires $(p+1)^d$ points around an interpolation point $\cp(x_i)$.
Furthermore, each of the points in the interpolation stencil must be advanced in time with a finite difference stencil. 
As a rule of thumb the diameter of the convolution of the interpolation and finite difference stencil gives a good value for the bandwidth.
More details on finding the optimal band are given in \cite[Appendix A]{cbm:icpm}.


\section{Three flows}

\subsection{Flow on a circular cylinder}
\subsection{Flow on a sphere}
\subsection{Flow on a torus}

\section{Discussion}

\newpage


\begin{appendix}


\section{Differentiation formulae}

Gradient operators:
\begin{subequations}
\begin{alignat}{3}
\nabla &= \sum_i \frac{e_i}{m_i} \pd{}{q_i}, \\
\nabla \cdot \mb{u} &= \frac{1}{m_1 m_2 m_3} \left[ \pd{}{q_1} ( m_2 m_3 u_1) + \pd{}{q_2} ( m_1 m_3 u_2) + \pd{}{q_3} ( m_1 m_2 u_3) \right] = 0, \\
&\qquad = \frac{1}{m_1 m_2 m_3} \sum_{i=0}^3 \pd{}{q_i} ( m_j m_k u_i), \label{div} \\
\nabla^2 &= \frac{1}{m_1 m_2 m_3} \left[ \pd{}{q_1} \left( \frac{m_2 m_3}{m_1} \pd{}{q_1}\right) 
+ \pd{}{q_2} \left( \frac{m_3 m_1}{m_2} \pd{}{q_2}\right) 
+ \pd{}{q_3} \left( \frac{m_1 m_2}{m_3} \pd{}{q_3}\right)\right], \\
&\qquad = \frac{1}{m_1 m_2 m_3} \sum_{i=0}^3  \pd{}{q_i} \left( \frac{m_j m_k}{m_i} \pd{}{q_i}\right).
\end{alignat}
\end{subequations}

\noindent Derivatives of unit vectors are \cite[p.483]{happel_book}
\begin{alignat}{3} 
\pd{\mb{e}_i}{x_j} &= \frac{\mb{e}_j}{m_i} \pd{m_j}{x_i}  \\ 
\pd{\mb{e}_i}{x_i} &= -\frac{\mb{e}_j}{m_j} \pd{m_i}{x_j} - 
\frac{\mb{e}_k}{m_k} \pd{m_i}{x_k}.
\end{alignat}

\noindent or simply \cite[p.241]{roy_2002}:
\begin{equation}
\pd{\mb{e}_i}{x_j} = \frac{\mb{e}_j}{m_i} \pd{m_j}{x_i} - \delta_{ij} \sum_{k=1}^3 \frac{\mb{e}_k}{m_k} \pd{m_i}{x_k}.
\end{equation}

\subsection{Surface quantities}
\begin{equation} \label{surfdiv}
\nabla_s \cdot (q_1 \ea + q_2 \eb) = \frac{1}{a_1 a_2} \left[ \pd{}{x_1} (a_2 q_1) + \pd{}{x_2} (a_1 q_2) \right].
\end{equation}

\begin{equation}
\nabla_s = \pd{}{x_1} \frac{\ea}{a_1} + \pd{}{x_2} \frac{\eb}{a_2}.
\end{equation}

\begin{equation}
\nabla_s^2 = \frac{1}{a_1 a_2} \left[ \pd{}{x_1} \left( \frac{a_2}{a_1} \pd{}{x_1}\right) + \pd{}{x_2} \left(\frac{a_1}{a_2} \pd{}{x_2}\right)\right].
\end{equation}

\section{Stress formula}

\begin{subequations} \label{stress}
\begin{align}
\sigma_{11} &= -p + \frac{2\mu}{m_1} \left[ \pd{u_1}{x_1} + \frac{u_2}{m_2} \pd{m_1}{x_1} + u_3 \pd{m_1}{n} \right] \\ 
\sigma_{22} &= -p + \frac{2\mu}{m_2} \left[ \pd{u_2}{x_2} + \frac{u_1}{m_1} \pd{m_2}{x_1} + u_3 \pd{m_2}{n} \right] \\ 
\sigma_{33} &= -p + 2\mu \pd{u_3}{n} \\
\sigma_{12} &= \frac{\mu}{m_1 m_2} \left[ m_1 \pd{u_1}{x_2} - \pd{m_1}{x_2} u_1 + m_2 \pd{u_2}{x_1} - \pd{m_2}{x_1} u_2 \right] \\ 
\sigma_{13} &= \frac{\mu}{m_1} \left[ m_1 \pd{u_1}{n} - u_1 \pd{m_1}{n} + \pd{u_3}{x_1} \right] \\ 
\sigma_{23} &= \frac{\mu}{m_2} \left[ m_2 \pd{u_2}{n} - u_2 \pd{m_2}{n} + \pd{u_3}{x_2} \right].
\end{align}
\end{subequations}




\section{Normal and tangents at the surface}

The surface is given by $n = h(x_1, x_2, t)$, or with $\mb{r}(x_1, x_2, n) = \mb{r}(x_1, x_2) + h(x_1, x_2, t) \ec$. The outwards unit normal vector must be 
\begin{equation} \label{surfnorm}
\mb{n} = \frac{\nabla (n-h)}{|\nabla (n-h)|} 
= \frac{\left[ -\frac{1}{m_1} \pd{h}{x_1} \ea - \frac{1}{m_2} \pd{h}{x_2} \eb + \ec\right] }{|\nabla (n-h)|}
= \frac{n_1 \ea + n_2 \eb + n_3 \ec}{\sqrt{n_1^2 + n_2^2 + n_3^2}}.
\end{equation}

\begin{multline} \label{curv}
\scr{K} = \frac{1}{\tilde{m_1} \tilde{m_2}} \left[ \pd{}{x_1} \left( \frac{\tilde{h}_2^2}{\scr{A}} \pd{h}{x_1}\right)
+ \pd{}{x_2} \left( \frac{\tilde{h}_2^2}{\scr{A}} \pd{h}{x_2}\right)\right] \\
+ \frac{1}{\scr{A}} \left[ \left(\tilde{h}_1^2 + \left(\pd{h}{x_1}\right)^2\right) \frac{a_2 \kappa_2}{\tilde{h}_1}
+ \left(\tilde{h}_2^2 + \left(\pd{h}{x_2}\right)^2\right) \frac{a_1 \kappa_1}{\tilde{h}_2}\right]
\end{multline}

\noindent where the surface area element is 
\begin{equation}
\scr{A} = \sqrt{\tilde{h}_1^2 \tilde{h}_2^2 + \tilde{h}_2^2 \left(\pd{h}{x_1}\right)^2 + \tilde{h}_1^2 \left(\pd{h}{x_2}\right)^2}.
\end{equation}

%In order to derive the tangential stress, we will need two vectors that span the tangent plane at the point $(x_1, x_2, n)$. If $\mb{t}$ is an arbitrary tangent, then $\mb{n} \cdot \mb{t} = 0$. This gives us two tangent vectors: 
%\begin{align}
%\mb{t}_1 
%&= \frac{n_2 \ea - n_1 \eb}{\sqrt{n_1^2 + n_2^2}} 
%= \frac{\left[-\frac{1}{m_2} \pd{h}{x_2} \ea + \frac{1}{m_1} \pd{h}{x_1} \eb\right]}
%{\sqrt{\left(\frac{1}{m_2} \pd{h}{x_2}\right)^2 + \left(\frac{1}{m_1} \pd{h}{x_1}\right)^2}}  \\
%\mb{t}_2 
%&= \frac{n_3 \ea - n_1 \ec}{\sqrt{n_1^2 + n_3^2}} 
%= \frac{\left[ \ea + \frac{1}{m_1} \pd{h}{x_1} \ec\right]}
%{\sqrt{1 + \left(\frac{1}{m_1} \pd{h}{x_1}\right)^2}} 
%\end{align}



%\noindent For the two tangents that span the tangent plane, recall that the position vector is given by $\mb{r} = x_1 \ea + x_2 \eb + n \ec$. For this, we need to use the formulae (\ref{}) to differentiate the unit vectors. We get
%\begin{gather}
%\pd{\ea}{x_1} = -\eb m_2 \pd{}{x_2} \left(\frac{1}{m_1}\right) - \ec \pd{}{n} \left(\frac{1}{m_1}\right) \\
%\pd{\ea}{x_2} = \eb m_1 \pd{}{x_1} \left(\frac{1}{m_2}\right) \\ 
%\pd{\ea}{x_3} = 0 \\ 
%\pd{\ec}{x_1} = \pd{\ec}{x_2} = 0, \quad \pd{\ec}{n} = 1 \\ 
%\end{gather}
%
%\begin{align}
%\pd{\mb{r}}{x_1}  &= \ea + x_1 \pd{\ea}{x_1} + x_2 \pd{\eb}{x_1} + \pd{h}{x_1} \ec  \\ 
%&= \ea\left[ 1 + x_2 m_2 \pd{m_1^{-1}}{x_2}\right] + \eb \left[ -x_1 m_2 \pd{m_1^{-1}}{x_2}\right]
%+ \ec \left[\pd{h}{x_1} - x_1 \pd{m_1^{-1}}{n}\right]. 
%\end{align}

\end{appendix}

\bibliographystyle{plain}
% \bibliographystyle{/Users/trinh/Dropbox/documents/bib/styles/apsrev4-1-edit}
\bibliography{/Users/trinh/Dropbox/documents/bib/philmaster}

\end{document}

